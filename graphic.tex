%
% FECHA: 2021-02-14
% AUTOR: Julio Alejandro Santos Corona
% CORREO: jualesac@yahoo.com
% TÍTULO: tabla.tex
%
% Descripción: Documentación de la clase GRAPHIC.
%

\documentclass[10pt]{article}
\usepackage[spanish]{babel}
\usepackage[utf8]{inputenc}
\usepackage{listings}
\usepackage{anysize}
\usepackage{colortbl}
\title{GRAPHIC v.1.0}
\author{Julio Alejandro Santos Corona}
\date{14 de febrero de 2021}

\marginsize{2.5cm}{2.5cm}{0cm}{1.5cm}

\definecolor{comment}{rgb}{0.5, 0.5, 0.5}
\definecolor{background}{rgb}{0.21, 0.24, 0.25}

\begin{document}
\lstdefinelanguage{JavaScript}{
	morekeywords={
		new,
		function,
		var,
		let,
		GEOMETRY,
		GRAPHIC,
		notationX,
		notationY,
		onBuild,
		setPoint,
		active,
		build,
		clear,
		setRange,
		getID,
		getPoints,
		getStatus,
		length,
		reset,
		create
	},
	sensitive=false,
	morecomment=[l][\color{comment}]{//},
	morecomment=[s][\color{comment}]{/*}{*/}
}

\lstset{breaklines=true, tabsize=4, language=JavaScript}
\lstset{basicstyle=\small, numbers=left, numberstyle=\small, stepnumber=1, numbersep=-12pt, backgroundcolor=\color{white}, frame=leftline}

\maketitle

La clase crea grápicas de puntos de forma simple, las gráficas pueden ser exploradas o segmentadas.

\section{Instancia}

La instancia se crea a través del constructor de la función \textbf{GEOMETYR.GRAPHIC.main} al que se le pasa un objeto como argumento.
\\
\begin{lstlisting}
	var graphic = new GEOMETRY.GRAPHIC.main (config);
\end{lstlisting}

\subsection{Configuración}

La configuración de la gráfica se realiza a través de un objeto pasado durante la creación de instancia con la siguiente estructura:
\\\\
\begin{tabular}{|m{2.2cm}|m{1.5cm}|m{11.5cm}|}
	\hline
	\rowcolor{black}\textcolor{white}{Propiedad} & \textcolor{white}{Tipo} & \textcolor{white}{Descripción} \\
	\hline
	width & number & Ancho de la gráfica. \\
	\hline
	height & number & Alto de la gráfica. \\
	\hline
	nameX & string & Nombre del eje x. \\
	\hline
	nameY & string & Nombre del eje y. \\
	\hline
	marginL & number & Margen izquierdo (configura el espacio dedicado al nombre del eje y). \\
	\hline
	marginR & number & Margen derecho. \\
	\hline
	maxUnitsX & number & Número máximo de segmentos calculados para el eje x. \\
	\hline
	maxUnitsY & number & Número máximo de segmentos calculados para el eje y. \\
	\hline
	directionX & string & Dirección del eje x ( + o - ). \\
	\hline
	directionY & string & Dirección del eje y ( + o - ). \\
	\hline
	notationX & function & Callback que toma el valor de escala y lo transforma según la función. \\
	\hline
	notationY & function & Callback que toma el valor de escala y lo transforma según la función. \\
	\hline
\end{tabular}
\\\\
\begin{lstlisting}
	var graphic = new GEOMETRY.GRAPHIC.main ({
		width: 1350,
        height: 580,
        nameX: "ops",
        nameY: "min",
        marginL: -30,
        marginR: -12,
        maxUnitsX: 13,
        maxUnitsY: 10,
        directionX: "+",
        directionY: "-",
        notationX: function (val) { return val; },
        notationY: function (val) { return val; }
	});
\end{lstlisting}
\newpage
\subsection{Métodos}

Al crear la instancia se contarán con los siguientes métodos:

\begin{itemize}
	\item \textbf{build} - Construcción de la gráfica, soporta un booleano como argumento para centrarse en un segmento o sólo construir dicho segmento.
	\item \textbf{clear} - Elimina las gráficas mostradas.
	\item \textbf{onBuild} - Recibe una función como argumento que será ejecutado en cada construcción de puntos.
	\item \textbf{create} - Clase que permite crear espacios geométricos.
\end{itemize}

\section{Espacios geométricos}

El espacio geométrico es un conjunto independiente de puntos que serán graficados.
\\\\
Dicho espacio es instanciado a partir de la clase \textbf{create} que requiere un objeto con la siguiente configuración.
\\\\
\noindent
\begin{tabular}{|m{2.2cm}|m{1.5cm}|m{11.5cm}|}
	\hline
	\rowcolor{black}\textcolor{white}{Propiedad} & \textcolor{white}{Tipo} & \textcolor{white}{Descripción} \\
	name & string & Nombre del espacio de puntos. \\
	\hline
	color & string & Color de línea en lenguaje css. \\
	\hline
	fill &
	object &
	\begin{tabular}{|m{2cm}|m{1.5cm}|m{6.7cm}|}
		width & number & Anchura del patrón de relleno. \\
		\hline
		height & number & Altura del patrón de relleno. \\
		\hline
		color & string & Color de relleno en formato css. \\
		\hline
		type & string & l = líneas, c = círculos \\
		\hline
		center & object & Activo sólo si type es c. \{ x: 0, y: 0 \} \\
		\hline
		line & string & Activo sólo si type es l. \\
	\end{tabular} \\
	\hline
\end{tabular}
\\\\
\begin{lstlisting}
	let g1 = new graphic.create ({
		name: "G1",
        color: "rgb(67, 124, 156)",
        fill: {
            width: 5,
            height: 5,
            color: "rgb(67, 124, 156)",
            type: "l",
            line: "M 5 0, 0 5"
        }
	});
\end{lstlisting}
\noindent
El objeto contiene los siguientes métodos:

\begin{itemize}
	\item \textbf{getID} - Arroja el id del espacio geométrico.
	\item \textbf{length} - Cantidad de puntos del espacio.
	\item \textbf{active} - Recibe un booleano para indicar si será graficado o no.
	\item \textbf{getStatus} - Indica si el espacio está activo o no.
	\item \textbf{setPoint} - Crea un nuevo punto en el espacio geométrico.
	\item \textbf{getPoints} - Obtiene los puntos pertenecientes al espacio geométrico.
	\item \textbf{setRange} - Indica un rango de puntos a mostrar.
	\item \textbf{reset} - Elimina todos los puntos 
\end{itemize}
\newpage
\section{Uso}

Para crear una gráfica se debe tener un espacio geométrico distinto del vacío.
\\\\
La forma de crear puntos se realiza a través del método \textbf{setPoint} con la siguiente estructura:
\\\\
\begin{tabular}{|m{2.2cm}|m{1.5cm}|m{1.5cm}|m{10cm}|}
	\hline
	\rowcolor{black}\textcolor{white}{Parámetro} & \textcolor{white}{Tipo} & \textcolor{white}{Opcional} & \textcolor{white}{Descripción} \\
	\hline
	x & number & no & Valor del eje x \\
	\hline
	y & number & no & Valor del eje y \\
	\hline
	i & object & si & Información adicional del punto (Se visualizarán en el cuadro de información). \\
	\hline
\end{tabular}
\noindent
\\\\
En caso de requerir un rango específico se puede utilizar un rango a través del método \textbf{setRange}.
\\\\
\begin{tabular}{|m{2.2cm}|m{1.5cm}|m{1.5cm}|m{10cm}|}
	\hline
	\rowcolor{black}\textcolor{white}{Parámetro} & \textcolor{white}{Tipo} & \textcolor{white}{Opcional} & \textcolor{white}{Descripción} \\
	\hline
	start & number & no & Número de punto inicial (empezando desde el 0) \\
	\hline
	end & number & si & Número de punto final, en caso de no declararlo se tomará el punto final \\
	\hline
\end{tabular}
\\\\
\begin{lstlisting}
	let g1 = new graphic.create (config);
	
	g1.setPoint (1, 6);
	g1.setPoint (6.1, 1);
	g1.setPoint (7.22, 9);
	g1.setPoint (8.9, 4.2);
	g1.setPoint (10.1, 3);
	g1.setPoint (15, 4.28);
	
	g1.setRange (3, 4); //Se seleccionan los puntos (8.9, 4.2) y (10.1, 3)
	
	g1.active (true);
\end{lstlisting}
\noindent
Para la construcción de la gráfica se utiliza el método \textbf{build} de la clase \textbf{GRAPHIC}
\\\\
\begin{tabular}{|m{2.2cm}|m{1.5cm}|m{1.5cm}|m{10cm}|}
	\hline
	\rowcolor{black}\textcolor{white}{Parámetro} & \textcolor{white}{Tipo} & \textcolor{white}{Opcional} & \textcolor{white}{Descripción} \\
	\hline
	range & bool & si & En caso de ser true se tomará el rango configurado permitiendo moverse por la gráfica, caso contrario sólo se mostrará dicho rango. \\
	\hline
\end{tabular}
\\\\
\begin{lstlisting}
	let graphic = new GEOMETRY.GRAPHIC.main (config);
	let g1 = new graphic.create (configGraphic);
	
	g1.setPoint (1, 6);
	g1.setPoint (6.1, 1);
	g1.setPoint (7.22, 9);
	g1.setPoint (8.9, 4.2);
	g1.setPoint (10.1, 3);
	g1.setPoint (15, 4.28);
	
	g1.setRange (3, 4); //Se seleccionan los puntos (8.9, 4.2) y (10.1, 3)
	
	g1.active (true);
	
	graphic.build (true);
\end{lstlisting}

\section{Extra}

\subsection{Funciones de notación}

Durante la instanciación de la clase GRAPHIC se incluye la propiedad \textbf{notationX} y \textbf{notationY}; dicha propiedad recibe como argumento el valor del punto y es útil cuando se quiere cambiar la representación de la escala.
\\\\
El ejemplo siguiente muestra cómo cambiar la escala a meses:
\\\\
\begin{lstlisting}
	function monthScale (value) {
		let months = [
			"ene",
			"feb",
			"mar",
			"abr",
			"may",
			"jun",
			"jul",
			"ago",
			"sep",
			"oct",
			"nov",
			"dic"
		];

        return months[Math.floor (value % 12)];
	}
\end{lstlisting}

\subsection{Validaciones extra}

La clase GRAPHIC también cuenta con el método \textbf{onBuild} que solicita como argumento una función que recibe de el objeto circle y los valores del punto, dicha función es ejecutada cada vez que se imprime un punto sobre la gráfica.
\\\\
A continuación se muestra un método que verifica cierto rango de valores en y, en caso de ser menores a 11 el punto cambiará su color a rojo.
\\
\begin{lstlisting}
	let graphic = new GEOMETY.GRAPHIC.main (config);
	
	graphic.onBuild (function (circle, point) {
		if (point.info.y < 11) {
            circle.setAttribute ("style", "stroke-width: 2px;");
            circle.setAttribute ("stroke", "rgb(254, 0, 0)");
        }
	});
\end{lstlisting}

\end{document}